\documentclass[letterpaper,landscape]{article}

\usepackage[spanish]{babel}
\usepackage[utf8]{inputenc}
\usepackage{longtable}
\usepackage[landscape]{geometry}
\usepackage{fancyhdr}
\usepackage{graphicx}


\setlength{\oddsidemargin}{-0.4in}		% default=0in
\setlength{\evensidemargin}{-0.4in}
\setlength{\textwidth}{9.8in}		% default=9in

\setlength{\columnsep}{0.5in}		% default=10pt
\setlength{\columnseprule}{1pt}		% default=0pt (no line)

\setlength{\textheight}{5.85in}		% default=5.15in
\setlength{\topmargin}{-0.15in}		% default=0.20in
\setlength{\headsep}{0.25in}		% default=0.35in

\setlength{\parskip}{1.2ex}
\setlength{\parindent}{0mm}

\pagestyle{fancy}

\lhead{\setlength{\unitlength}{0mm}
  \begin{picture}(10,10)
    \put(0,0){\includegraphics[width=1.5cm]{logo.jpg}}
  \end{picture}}
%\lhead{\bfseries{Cálculo Integral I}}

  %PARA QUE NO SALGA EL NUMERO DE PAGINA
\cfoot{}



\begin{document}
\begin{center}
\textsc{{\large Universidad Nacional Autónoma de Honduras -- Facultad de Ciencias -- Departamento de Matemáticas}\\ Listado de Notas Originales}
\end{center}
\begin{flushleft}
CODÍGO: \underline{ \hspace{5pt}    \texttt  \hspace{5pt}   }
NOMBRE DE LA ASIGNATURA: \underline{    \hspace{5pt} \texttt  \hspace{5pt}   } 
SECCIÓN: \underline{\hspace{5pt}    \texttt{1201}  \hspace{5pt}}
UV: \underline{\hspace{5pt} \texttt{4} \hspace{5pt}}\\
PERIODO: \underline{\hspace{5pt}    \texttt  \hspace{5pt}}
AÑO: \underline{\hspace{5pt}    \texttt{2015}  \hspace{5pt}}
PROFESOR: \underline{\hspace{5pt}    \texttt  \hspace{5pt}}
NÚMERO DE ASIGNATURA: \underline{\hspace{5pt}    \texttt{77849}  \hspace{5pt}}\\
\end{flushleft}
%%% AQUI PONEMOS LA TABLA
%\begin{center}
\begin{longtable}[c]{|c|c|l|l|l|l|c|c|c|c|c|c|c|c|c|c|}
\hline
\input{notas.cvs}
\end{longtable}
%\end{center}
\vspace{25pt}
\begin{flushright}
\begin{minipage}[b]{10cm}
\hrule
\vspace{3pt}
\begin{center} Luis Berlioz \end{center}
\end{minipage}
\end{flushright}

\end{document}
